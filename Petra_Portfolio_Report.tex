\documentclass[11pt, a4paper]{article}
\usepackage[utf8]{inputenc}
\usepackage{geometry}
\usepackage{graphicx}
\usepackage{listings}
\usepackage{xcolor}
\usepackage{hyperref}
\usepackage{bookmark}
\usepackage{amsmath}
\usepackage{float}

\geometry{margin=1in}

% Code snippet styling
\definecolor{codegreen}{rgb}{0,0.6,0}
\definecolor{codegray}{rgb}{0.5,0.5,0.5}
\definecolor{codepurple}{rgb}{0.58,0,0.82}
\definecolor{backcolour}{rgb}{0.95,0.95,0.92}

\lstdefinestyle{mystyle}{
    backgroundcolor=\color{backcolour},   
    commentstyle=\color{codegreen},
    keywordstyle=\color{magenta},
    numberstyle=\tiny\color{codegray},
    stringstyle=\color{codepurple},
    basicstyle=\ttfamily\footnotesize,
    breakatwhitespace=false,         
    breaklines=true,                 
    captionpos=b,                    
    keepspaces=true,                 
    numbers=left,                    
    numbersep=5pt,                  
    showspaces=false,                
    showstringspaces=false,
    showtabs=false,                  
    tabsize=2,
    language=C++
}

\lstset{style=mystyle}

\title{\textbf{Project Petra: High-Frequency Computational Finance Framework} \\ \large Technical Portfolio Report}
\author{Kenny Yu}
\date{\today}

\begin{document}

\maketitle

\begin{abstract}
This report details the architecture, implementation, and performance characteristics of \textbf{Petra}, a next-generation high-frequency C++20 framework for financial engineering. The project demonstrates advanced capabilities in low-latency systems, mathematical optimization, and systemic risk management. Key achievements include a zero-allocation pricing engine achieving 71 million simulation paths per second and a toxicity-aware market making system designed to enhance financial market stability.
\end{abstract}

\tableofcontents
\newpage

\section{Executive Summary}
\textbf{Petra} is a dual-component financial system refactored from legacy architectures to modern C++20 standards. It addresses critical needs in the US financial infrastructure for robust, high-speed, and fail-safe trading technologies.

The system is composed of two primary modules:
\begin{enumerate}
    \item \textbf{GreekCore}: A pure, header-only template library for quantitative finance (Yield Curves, Option Pricing, Root Finding).
    \item \textbf{MarketMaker}: An event-driven, low-latency execution engine with real-time risk controls.
\end{enumerate}

\section{Technical Architecture}

\subsection{Separation of Concerns}
To ensure systemic resilience, the calculation layer is strictly decoupled from the execution layer.
\begin{itemize}
    \item \textbf{Quant Layer (GreekCore)}: Responsible for the "Truth" (Fair Value). Uses \texttt{double} precision math and focuses on numerical stability.
    \item \textbf{Execution Layer (MarketMaker)}: Responsible for the "Action" (Orders). Focuses on nanosecond-level latency and state management.
\end{itemize}
This architecture prevents execution errors (e.g., algorithm crashes) from corrupting the fundamental pricing mathematics.

\subsection{C++20 Implementation Details}
The codebase strictly adheres to modern C++ standards to maximize compile-time optimization and memory safety:
\begin{itemize}
    \item \textbf{Concepts}: Used to enforce interfaces (e.g., \texttt{PayOffConcept}, \texttt{MakingStrategy}) without the runtime overhead of virtual functions (v-tables).
    \item \textbf{Zero-Allocation}: Critical hot paths (Monte Carlo simulations, Strategy Ticks) permit zero heap allocations, utilizing stack memory to prevent garbage collection pauses and memory fragmentation.
    \item \textbf{Data-Oriented Design}: Structs are packed to fit within CPU cache lines (L1/L2) to minimize memory access latency.
\end{itemize}

\section{Core Modules}

\subsection{Optimized Monte Carlo Pricer}
The pricing engine replaces standard library generators with high-performance alternatives to achieve sub-microsecond latency.
\begin{itemize}
    \item \textbf{RNG}: Replaced \texttt{std::mt19937} (2.5KB state) with \textbf{Xoshiro256++} (32-byte state), fitting entirely in CPU registers.
    \item \textbf{Distribution}: Implemented a manual, branchless \textbf{Box-Muller Transform} to generate normally distributed random numbers, bypassing the overhead of \texttt{std::normal\_distribution}.
    \item \textbf{Vectorization}: The processing loop handles 4 paths per iteration (2 random $\times$ 2 antithetic variates) to maximize instruction pipelining.
\end{itemize}

\subsection{Market Toxicity Detection (VPIN)}
A real-time risk module monitors order flow toxicity using the \textbf{Volume-Synchronized Probability of Informed Trading (VPIN)} metric.
\[
\text{Toxicity} = \frac{|V_{buy} - V_{sell}|}{V_{total}}
\]
The engine automatically widens spreads during toxic flow and triggers a \textbf{Circuit Breaker} to pull quotes when toxicity exceeds critical thresholds (0.8), preventing adverse selection and contributing to market stability.

\section{Performance Benchmarks}

Benchmarks were conducted on a linux environment using Google Benchmark.

\subsection{Throughput Results}
\begin{itemize}
    \item \textbf{Baseline}: $\approx$ 21 Million paths/second (Debug/Unoptimized).
    \item \textbf{Optimized}: $\textbf{71.02 Million paths/second}$ (Release, Xoshiro256, Box-Muller).
\end{itemize}
This represents a \textbf{3.4x} improvement over the baseline and enables real-time pricing of complex derivatives on consumer-grade hardware.

\subsection{Convergence Verification}
The Monte Carlo engine was mathematically verified to converge to the Black-Scholes analytical solution at a rate of $O(1/\sqrt{N})$, confirming the numerical correctness of the optimizations.

\section{National Interest Context}
This project demonstrates advanced expertise in \textbf{High-Frequency Trading Systems}, a field of critical importance to US financial leadership.
\begin{itemize}
    \item \textbf{Systemic Stability}: The toxicity detection and circuit breaker mechanisms directly address the prevention of "Flash Crashes."
    \item \textbf{Technological Modernization}: The refactoring from legacy Java to optimized C++20 provides a blueprint for modernizing institutional financial software.
\end{itemize}

\end{document}
